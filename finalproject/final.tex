\documentclass[11pt]{article}
\usepackage{enumerate}
\usepackage{setspace}
\usepackage{url}


\begin{document}

\author{Jayme Woogerd}
\title{Biocryptography \\ {\large The Future of User Authentication?}}
\maketitle

\section*{Abstract}
In traditional cryptography, authentication relies on some sort of shared knowledge,
usually a secret password or token. In today's wired world, these types of password
authentication systems are pervasive -- however, they face several fundamental
problems, chief of which that they cannot distinguish between genuine users and
attackers using stolen credentials. Biometric systems solve these problems by 
using physical characteristics to uniquely identify genuine users. However, 
these systems come with their unique set of vulnerabilities.

At the intersection of biometrics and traditional cryptography 
lies biocryptography, which fuses the strengths of both types of 
authentication systems. This project explores the state-of-the-art strategies 
in biocryptography to provide systems that are both accurate and robust to attacks.

\doublespacing
\section{Introduction}
In traditional cryptography, authentication relies on some sort of shared knowledge,
usually a secret password or token. In today's wired world, these types of password
authentication systems are pervasive -- however, they face several fundamental
problems, chief of which that they cannot distinguish between genuine users and
attackers using stolen credentials. Additionally, users are forced to manage 
multiple accounts with multiple passwords.  \\
Biometric systems solve these problems by using characteristics like fingerprints,
irises, even ear shape to uniquely identify genuine users. However, these
systems come with their unique set of vulnerabilities: what happens if a user's
biometric data is compromised? A password can be easily reset...but a fingerprint
certainly cannot. At the intersection of biometrics and traditional cryptography
lies biocryptography, which fuses the strengths of both types of authentication systems.
This paper explores the state-of-the-art strategies in biocryptography to
provide systems that accurately authenticate genuine users and are robust to 
traditional biometric attacks.

\section{To the Community}
According to the results of an online registration and password study conducted 
by Janrain in 2012, 58 percent of adults have five or more unique 
passwords associated with their online logins and 30 percent of people have 
more than 10 unique passwords they need to remember. Additionally, almost 2 in 
5 (37\%) have to ask for assistance on their user name or password for at least 
one website per month \cite{Janrain12}.

The current models for user authentication have created a ridiculous situation:
on the one hand, experts implore us that a password should be long, random (and 
therefore not memorable) and unique so as to be resistant to cracking, but on
the other hand, we have to create, keep track of, and periodically change 
credentials for \textit{every single} service we sign up for. Anecdotally, for 
most people convenience trumps security and they end up using (and reusing) 
weak passwords over and over again.

    The problem with password breaches is only going to get worse

Thus, the original motivation for this project was my own frustration with the
current state of user authentication systems. It's deeply unsatisfying that so 
much of the onus is on the user to manage and secure her own credentials. Therefore, this project explores an alternative to a password-based 
authentication system: biometric systems and the biocrytographic methods used 
to secure them. I wanted to know if biometric systems as seen in popular 
culture (e.g. \textit{Minority Report}) are even feasible and if so, in what 
ways are they better than current traditional authentication systems.
\section{Traditional Cryptography and Authentication}
The basic idea in cryptography is to allow two entities to send and receive 
messages to each other securely and confidentially in the presence of a 
third-party, or adversary. This is done through the process of data 
\textit{encryption}, that is, transforming the message into an unreadable form 
to anyone who does not know how to decrypt it. In general, there are four 
main goals of modern cryptography: 1) confidentiality, 2) data integrity, 
3) non-repudiation, and 4) authentication \cite{Biocryptography10}.

% In modern cryptography, there are 
% two main flavors of encryption algorithms: symmetric- or private-key 
% encryption and asymmetric or public-key encryption.
    \subsection{Authentication Systems}
    \subsection{Shortcomings}
        \begin{enumerate}[1.]
        \item Knowledge such as passwords and PINs can be easily forgotten (some stats about the average number of passwords people have to manage?)
        \item Passwords and PINs can be guessed using social engineering or dictionary/wordlist attacks (stats about breaches, etc?)
        \item Tokens like key or cards can be stolen or misplaced
        \end{enumerate}
\section{Biometrics}
Rather than relying on a shared secret or key, in biometric systems, 
authentication relies on the physiological or behavior features of a person. 
Genuine users are recognized using characteristics like fingerprints,
irises, or even ear shape. Biometric systems can be more reliable than traditional
password-based systems because biometric features cannot be lost or forgotten
and they are difficult to copy or forge and to share or distribute \cite{Fingerprint07}. 
\subsection{Fingerprint Systems}
Fingerprint systems are the oldest and most commonly used systems in biometrics.
\subsection{Shortcomings}
        \begin{enumerate}[1.]
        \item Accuracy, security, and privacy challenges
        \item Biometric system attacks
        \end{enumerate}
\section{Biocryptographic Methods and Applications}
Because of the characteristics of fingerprint images, traditional methods like 
DES and RSA cannot be used for encryption. There characteristics include 
bulk data capacity and high correlation among pixels.
    \subsection{General overview and template protection}
    \subsection{Fingerprint Fuzzy-Vault Algorithm}
\section{Conclusion}

\singlespace
\nocite{*}

\bibliography{biblio}{}
\bibliographystyle{apalike}

\end{document}
