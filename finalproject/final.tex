\documentclass[11pt]{article}
\usepackage{setspace}
\usepackage{url}
\usepackage[shortlabels]{enumitem}


\begin{document}

\author{Jayme Woogerd}
\title{Biocryptography \\ {\large The Future of User Authentication?}}
\maketitle

\section*{Abstract}
In traditional cryptography, authentication relies on some sort of shared knowledge,
usually a secret password or token. In today's wired world, these types of password
authentication systems are pervasive -- however, they face several fundamental
problems, chief of which that they cannot distinguish between genuine users and
attackers using stolen credentials. Biometric systems solve these problems by 
using physical characteristics to uniquely identify genuine users. However, 
these systems come with their unique set of vulnerabilities.

At the intersection of biometrics and traditional cryptography lies 
biocryptography, the study of cryptographic methods for securing biometric 
systems. This project explores the state-of-the-art strategies 
in biocryptography to provide systems that are both accurate and robust to attacks.

\doublespacing
\section{Introduction}
In traditional cryptography, authentication relies on some sort of shared knowledge,
usually a secret password or token. In today's wired world, these types of password
authentication systems are pervasive -- however, they face several fundamental
problems, chief of which that they cannot distinguish between genuine users and
attackers using stolen credentials. Additionally, users are forced to manage 
multiple accounts with multiple passwords.  \\
Biometric systems solve these problems by using characteristics like fingerprints,
irises, even ear shape to uniquely identify genuine users. However, these
systems come with their unique set of vulnerabilities: what happens if a user's
biometric data is compromised? A password can be easily reset...but a fingerprint
certainly cannot. At the intersection of biometrics and traditional cryptography
lies biocryptography, the study of specialized cryptographic methods for securing 
biometrics systems. This paper explores the state-of-the-art strategies in 
biocryptography to provide systems that accurately authenticate genuine users 
and are robust to traditional biometric attacks.

\section{To the Community}
According to the results of an online registration and password study conducted 
by Janrain in 2012, 58 percent of adults have five or more unique 
passwords associated with their online logins and 30 percent of people have 
more than ten unique passwords they need to remember. Additionally, almost 2 in 
5 (37\%) have to ask for assistance on their user name or password for at least 
one website per month \cite{Janrain12}.

The current models for user authentication have created a ridiculous situation:
on the one hand, experts implore us that a password should be long, random (and 
therefore not memorable) and unique so as to be resistant to cracking, but on
the other hand, we have to create, keep track of, and periodically change 
credentials for \textit{every single} service we sign up for. Anecdotally, for 
most people convenience trumps security and they end up using (and reusing) 
weak passwords over and over again.

The situation is exacerbated by the fact that the entities we trust to secure
our passwords are not secure themselves: if anything it seems the frequency with
which we are hearing about major security breaches (e.g. LinkedIn, Target, Sony)
is increasing. From the LinkedIn breach on June 5, 2012 alone, hackers 
gained access to 6.5 million user passwords -- and had cracked and posted 
in cleartext more than 60\% of the unique passwords by the next 
day \cite{ComputerWorld}.

Thus, the original motivation for this project was my own frustration with the
current state of user authentication systems. It's deeply unsatisfying that so 
much of the onus is on the user to manage and secure her own credentials. Therefore, this project explores an alternative to a password-based 
authentication system: biometric systems and the biocrytographic methods used 
to secure them. I wanted to know if biometric systems as seen in popular 
culture (e.g. \textit{Minority Report}) are even feasible and if so, in what 
ways are they better than current traditional authentication systems.

\section{Cryptography}
In cryptography, the basic idea is to allow two entities to send and receive 
messages to each other securely and confidentially in the presence of a 
third-party, or \textit{adversary}. This is done through the process of data 
\textit{encryption}, that is, transforming the message into an unreadable form 
to anyone who does not know how to decrypt it. Traditionally, the 
cryptographic process involves using an \textit{encryption algorithm} and 
a \textit{cryptographic key} to turn an original message in \textit{plaintext}
into a scrambled, unreadable one, i.e. in \textit{cypertext}. 

\subsection{Encryption Algorithms}
In modern cryptography, there are two main flavors of encryption algorithms: 
symmetric- or private-key encryption and asymmetric or public-key encryption.
Symmetric-key encryption algorithms use the same cryptographic key and algorithm to
both encrypt plaintext into cyphertext and decrypt the cyphertext back into
plaintext. Among the most widely used symmetric-key algorithm is the Data 
Encryption Standard (DES). Asymmetric-key systems involve pairs of keys, which
are generated together: a public key for encryption and a private key for 
decryption. As the name suggests, the public key is shared publicly and can
be used by someone else to encrypt data to send to you. Data encrypted with this 
public key can only be decrypted using the private key, which you keep secret.
A common symmetric-key algorithm is RSA, named for its creators Rivest, 
Shamir, and Adleman \cite{Biocryptography10}.

\section{Authentication Systems}
In general, there are four main goals of modern cryptography: 1) 
confidentiality, 2) data integrity, 3) non-repudiation, and 4) 
authentication \cite{Biocryptography10}. The last goal, authentication, concerns
itself with verifying claims of identity. In the context of sending a message, 
the sender and receiver should be able to confirm the other's identity and 
the origin of the message. Authentication is distinguished from 
\textit{authorization}, which is the process of giving a party access to a
system or data based on the confirmation of their identity.

In a knowledge-based authentication systems, a user's identity is verified 
via a piece of knowledge: a password, pass phrase, or a personal 
identification number (PIN). However, there are several shortcomings to this
method: 1) As illustrated above, knowledge such as passwords and PINs can be 
easily forgotten, 2) Passwords and PINs can be guessed using social engineering
3) Even encrypted passwords are easily cracked with brute-forced and/or 
wordlist attacks 4) Passwords and PINs are easy to distribute and share in 
plaintext and 5) A password-based system cannot distinguish a genuine user from
an attacker using stolen or forged credentials.

\subsection{Biometric Authentication Systems}
Rather than relying on a shared secret or key, in biometric systems, 
authentication relies on the physiological or behavior features of a person. 
Genuine users are recognized using characteristics like fingerprints,
irises, or even ear shape \cite{Biocryptography10}. 


In general, biometric systems comprise of five separate elements. A 
\textit{biometric sensor} reads in the biometric information and usually 
does some quality checking on the sample, e.g. a fingerprint reader. The 
\textit{feature extractor} consumes this raw biometric information and pulls 
out a relevant feature set (or template) that represents the data. In a 
fingerprint system, these features would include minutiae details that 
characterize the fingerprint. The \textit{matcher} or \textit{matching model}
matches this sample template against a pre-stored template and comes up with
a score, i.e. the extent to which the sample matches the pre-stored template. Finally, it is common for systems to use a \textit{database} to store known 
templates.

Biometric systems can be more reliable than traditional
password-based systems because biometric features cannot be lost or forgotten
and they are difficult to copy or forge and to share or distribute \cite{Fingerprint07}. However, they are not without their own set of issues including 
those of accuracy (i.e. false positive and negative matches), security (e.g.
unlike a password it is impossible to replace a person's stolen biometric information), and privacy. Moreover, biometric systems are not impervious to 
attack, a biometric system comes with its own set of vulnerabilities.

\subsection{Biometric System Attacks}
Biometric systems come with their own set of unique attack vectors; according
to Ratha et al, they can be categorized into eight types \cite{Ratha01ananalysis}:
        \begin{enumerate}[1., topsep=0pt]
    \singlespace
        \item Fake biometric: attacker uses a reproduction of the biometric, e.g. a fake fingerprint
        \item Replay attack: attacker replays an old recorded signal, e.g. presents old copy of fingerprint image to the system
        \item Override feature extract: Attacker plants a Trojan horse in 
        the feature extractor so that it produces a feature set that the 
        attack chose
        \item Override matcher: Attacker tampers with matcher to produce
        artificially high or low match rates
        \item Tamper with the feature representation: Attacker replaces
        extracted feature set with a different, synthesized one
        \item Tamper with stored templates: Attacker tampers with database
        holding the templates, e.g modifies a template to result in fraudulent 
        authorization for an individual 
        \item Attack communication channel: Attack tampers with the templates 
        when the are enroute from storage database to the matcher
        \item Decision override: Recognition system works as expected but
        attacker changes final authentication decision at the last step 
        \end{enumerate}
    \doublespace

\section{Biocryptographic Methods and Applications}
According to Xi and Hu, among the types of attacks against biometric systems,
those targeting templates can be hard to detect and can cause the most damage.
\cite{Biocryptography10} Therefore, for a system to be secure, biometric 
templates should always be encrypted, both when stored and during 
the matching process. However, because of the characteristics of biometric 
data, traditional methods that use non-smooth functions, like DES and RSA, 
cannot be used for encryption \cite{Fingerprint07}. For example, given a 
feature set derived from a fingerprint, even tiny variances in the plaintext 
feature set will produce wildly different encrypted feature sets, making 
it impossible to do feature matching using encrypted templates.

\subsection{Template Encryption}
In general, to encrypt a template, \textit{T} we use a secret key, \textit{$K_E$}
and an encryption algorithm, \textit{E}, such that the encrypted version, 
\textit{C} is given by: \\
\indent C = E (T, $K_E$)
\\ Then, to decrypt, we apply a decryption algorithm, \textit{D} to \textit{C} 
and a decryption key \textit{$K_D$} to get the template back: \\
\indent T = D (C, $K_D$).
\\One biocryptographic technique is called key binding, in which the secret key
and the biometric data (i.e. the template) are combined to produce an artifact
in which both the template and the key are hidden. This artifact can then be
shared publicly since it is computationally infeasible to decrypt the artifact
directly \cite{Biocryptography10}.

\subsubsection{Fingerprint Fuzzy Vault}
A key-binding scheme proposed by Juels and Sudan is called the fuzzy vault 
algorithm \cite{Juels:2006:FVS:1110940.1110956}. This algorithm is especially
applicable to biometric data because it is \textit{error-tolerant} and 
invariant to order. That is, data can be encoded using a set of values (e.g. a
biometric feature set), and then unlocked with a \textit{different} set of 
values as long as there is some threshold of overlap between the sets. Order
invariance means that it does not matter which of the sets is used for 
locking and which is used for unlocking.

The basic algorithm works as follows: given a secret key and a template (i.e.
feature set), first encode the key as the coefficients of a polynomial function,
\textit{p(x)}. Apply \textit{p(x)} to every value in the feature set to 
produce a set of points that genuinely describes the polynomial. Then, to obscure
the template data, generate a "chaff" set, points that are within the domain
and range but do not lie on the polynomial. Finally, combine the two sets and 
scramble the order -- this set of points is the "fuzzy vault".
To decrypt the fuzzy vault, use the feature set provided by a user. If enough
of the points match the set used to encode the data within a given error, the
polynomial can be reconstructed and thus the secret key revealed \cite{Biocryptography10,Juels:2006:FVS:1110940.1110956}. 

As a proof of concept, I have implemented a 
simple "biometric" authentication system using the fuzzy vault
algorithm to encrypt identity/fingerprint pairs. The
code is available publicly at \url{https://github.com/jwoogerd/fuzzy_vault}.

\section{Conclusion}
Password-based authentication systems, though pervasive, are problematic in
that users must manage many sets of credentials, creating an incentive to use
and reuse weak, easy to crack passwords. Furthermore, a password-based system
cannot distinguish between a genuine user and an attacker using stolen credentials. Biometric systems solve these problems by using physiological characteristics 
to uniquely identify genuine users, but come with their own set of challenges
and vulnerabilities. Since many traditional cryptographic methods are unsuited to biometric data, the field of biocryptography explores the specialized 
cryptographic methods for securing biometrics systems.
\singlespace
\nocite{*}

\bibliography{biblio}{}
\bibliographystyle{apalike}

\end{document}
