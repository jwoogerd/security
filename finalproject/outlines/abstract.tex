\documentclass[11pt]{article}
\usepackage{fullpage}
\usepackage{enumerate}

\begin{document}
Jayme Woogerd \\
\indent Comp 116 - Computer Security \\
\indent October 14, 2014 \\
\indent Final Project Abstract \\

\section*{Biocryptography: The Future of User Authentication?}
In traditional cryptography, authentication relies on some sort of shared knowledge,
usually a secret password or token. In today's wired world, these types of password
authentication systems are pervasive -- however, they face several fundamental
problems, chief of which that they cannot distinguish between genuine users and
attackers using stolen credentials. Additionally, users are forced to manage 
multiple accounts with multiple passwords.  \\
Biometric systems solve these problems by using characteristics like fingerprints,
irises, even ear shape to uniquely identify genuine users. However, these
systems come with their unique set of vulnerabilities: what happens if a user's
biometric data is compromised? A password can be easily reset...but a fingerprint
certainly cannot. At the intersection of biometrics and traditional cryptography
lies biocryptography, which fuses the strengths of both types of authentication systems.
This project will explore the state-of-the-art strategies in biocryptography to
provide systems that accurately authenticate genuine users and are less 
vulnerable to traditional biometric attacks.

\end{document}